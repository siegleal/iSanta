
\subsection{Product}
\begin{description}
	\item[Memory Usage] Memory usage is extremely important in an iOS environment.  Apple strictly regulates memory consumption and app performance.  This metric can be measured using the iOS development kit.
	\item[Errors/KLOC] Tracking the number of errors is to measure the quality of written code.  Since the team will be porting another program to suit the new mobile requirements, there will be unforeseen errors due to the nature of moving to a smaller, more controlled platform.  Errors found will be reported in the backlog as bugs and the current KLOC can be associated with each bug to assess the metric.
\end{description}

\subsection{Communication}
\begin{itemize}
    \item The development team will be conveying progress though the tracking software Pivotal Tracker.  The tracker will present the client with current tasks being performed as well as completed tasks and deliverables.
    \item The development team will communicate intentions through a confidence document which rates each feature/need and conveys the confidence of completion of each feature as described by the development team.  This document will be updated as the outlook of each feature changes.

\subsection{Process}
	\item[Milestones] Using planned milestones for large parts of the program such as porting and working prototypes are good for planning ahead and reporting large parts of the project as complete.  Milestones will be the collection of many releases related to a single unified feature.
	\item[Backlog Tasks] This metric is critical in assessing the overall status of the project.  The amount of tasks in the backlog and the estimated time to complete the backlog as well as comparing the previous state to past states is a good indicator of how effective the current velocity of sprints are.  If the amount of work remaining is not gradually declining then a larger velocity should be taken in the next sprint to adjust.
	\item[Unmanaged Risks]  Tracking unmanaged risks for a given cycle will allow the team to prioritize the features in the backlog and address the riskier tasks first as to avoid schedule delays.  A reassessment of unmanaged risks will occur with each sprint as tasks are completed and risks are handled.
