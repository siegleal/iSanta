\subsection{Dates and Times}
\begin{itemize}
\item Dates will be written in accordance with military guidelines (DD Month YYYY).
\item Times will be written in 24-hour style.
\end{itemize}
\subsection{Version Control}
\paragraph{} We will be using GitHub to manage our files for this project. It will be a public repository that anyone can fork and use. We will have two main folders in the file system. One for our code and one for our documentation. The documents will be written in \LaTeX. The documents will be uploaded by the person who made the most recent edits. Our code will be written in a number of languages including C, C++ and Objective-C. The code files along with the project file will be uploaded to the GitHub Repository by the team member who made the last edit/addition.
\subsection{Task Tracking}
\paragraph{} We will be using Pivotal Tracker, by Pivotal Labs, to manage our sprints. Pivotal Tracker is a feature/ticket/bug-tracking web application built specifically for teams using agile project management methods. We are also using Pivotal as a means to communicate with our client: if he thinks of features outside of client meetings, he can immediately post a feature in Pivotal Tracker’s icebox. We can do the same. During meetings, we will review current items, mark items completed as necessary, then move items from the icebox to the next sprint’s list of items to work on. Finally, we’ll add an estimate of the rough amount of time a feature will take, on a scale of [0,1,2,3,5,8]. 

\subsection{Change Control}
\paragraph{}Changes will be handled by uploading requests to Pivotal Tracker, then discussing them during our weekly meetings. We believe strongly that all changes have a cost, and therefore that they should be tracked just as all other features are tracked. 

Once we finish the basic port, our software will have ample flexibility in accommodating additional features. We will discuss specific features to implement as Winter quarter begins and as we finalize the port. Both we and our client will add features to our backlog as we think of them. During meetings, we will prioritize features and discuss what might be suitable for implementation during the upcoming week. This flexible cycle allows us to avoid locking ourselves into specific design decisions until they actually need to be made. 

Both we and our clients will want to receive extensive feedback from in-field tests before we prioritize certain additional features to add. Our scope isn’t completely defined, so we’ll be able to put much more detail into this as the course progresses. 
