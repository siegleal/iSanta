\section{Risks}
\paragraph{}In this section we will be describing the risks that we will have to undertake during the course of this project. Next to each risk will be a severity rating and an impact that will either be Low, Medium, and High. Low means that the risk has a very small chance of occurring and if it does occur it will not have a noticeable impact. Medium means that the risk has a pretty good chance of occurring and would have a visible impact on the project, but we can probably structure our design to avoid it. High means that we will almost certainly run into this problem during the course of the project and if it happens it will require a significant amount of effort to develop a solution to over come it.

\subsection{Image Recognition Efficiency}
\paragraph{}\textsc{Risk:} Medium
\paragraph{}\textsc{Impact:} High
\paragraph{} This is a risk that if not accounted for early on will cause us big problems later. Our concern is that the algorithm we were given will not run efficiently on an iOS device. The algorithm may not normalize and scale the image which could cause inefficient or inaccurate results. If this happens then you will not be able to use the app effectively or at all.
\\
\\Solution: If we find that the algorithm isn't efficient enough then we will work to make it more efficient and possibly sacrifice some of the automation in favor of manually marking the bullet holes.

\subsection{Usability}
\paragraph{}\textsc{Risk:} Low
\paragraph{}\textsc{Impact:} Medium
\paragraph{} As the end users of our app will not necessarily be computer experts we need to make sure that the system is easy to learn and easy to use. If it is not usable then the app will probably not be used when testing weapons.
\\
\\Solution: We can avoid this risk by designing our app to be as straight forward as possible and present our clients with prototyped designs in order to get feedback and make changes.
